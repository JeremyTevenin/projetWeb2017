\documentclass[11pt, openany]{report}
\usepackage[utf8]{inputenc}
\usepackage[T1]{fontenc}
\usepackage[a4paper,left=2cm,right=2cm,top=2cm,bottom=2cm]{geometry}
\usepackage[frenchb]{babel}
\usepackage{libertine}
\usepackage[pdftex]{graphicx}
\usepackage{geometry}
\usepackage[pdftex]{graphicx}
\usepackage{adjustbox}
\usepackage{color}
\usepackage{setspace}
\usepackage{hyperref}
\usepackage{comment}
\usepackage{fancyhdr}
\pagestyle{fancy}

\setlength{\parindent}{0cm}
\setlength{\parskip}{1ex plus 0.5ex minus 0.2ex}
\newcommand{\hsp}{\hspace{20pt}}
\newcommand{\HRule}{\rule{\linewidth}{0.5mm}}

\fancyhead[L]{}
\fancyhead[C]{\textbf{Le Petit Scientifique}}
\fancyhead[R]{\includegraphics[width=2.5cm]{images/universite-rouen.jpg}}

\begin{document}

\begin{titlepage}
  \begin{sffamily}
  \begin{center}

    % Upper part of the page. The '~' is needed because \\
    % only works if a paragraph has started.

    \textsc{\LARGE UFR des Sciences et Techniques}\\[2cm]

    \textsc{\Large Rapport Mini projet LW1 – sCiMS}\\[1.5cm]

    % Title
    \HRule \\[0.4cm]
    { \huge \bfseries Le Petit Scientifique\\[0.4cm] }

	\HRule \\[2cm]
    \includegraphics[scale=0.2]{images/universite-rouen.jpg}
    \\[2cm]

    % Author and supervisor
    \begin{minipage}{0.4\textwidth}
      \begin{flushleft} \large
        Amine \textsc{BOUAZIZ}\\
        Jérémy \textsc{TEVENIN}\\
      \end{flushleft}
    \end{minipage}
    \begin{minipage}{0.4\textwidth}
      \begin{flushright} \large
        \emph{Rendu à :} M. \textsc{Mallet}\\
      \end{flushright}
    \end{minipage}

    \vfill

    % Bottom of the page
    {\large 1\ier{} 20 décembre 2016}

  \end{center}
  \end{sffamily}
\end{titlepage}

\newpage
%Table of contents
\tableofcontents

\newpage
\section{Introduction}
Le projet consiste en l'écriture d'un mini CMS pour la mise en ligne d'articles à caractère scientifique.
Le site sera divisé en catégories, sous-catégorie et articles pour organiser le contenu.
Il y a 3 types d'utilisateurs ~: 
- l'administrateur qui peut gérer les catégories, sous-catégories et rédacteurs
- les rédacteurs peuvent ajouter des articles et ils peuvent aussi modifier ou supprimer leur propre articles 
- les simple utilisateur peuvent juste visionner les articles

\newpage
\section{Manuel d'utilisation}

\newpage
\section{Conception}

\newpage
\section{Les bases de données}


\newpage
\section{Inscription/connexion}


\newpage
\section{Administration du site}
\subsection{Par l'administrateur}

\subsection{Par les rédacteurs}


\newpage
\section{Le style du site (CSS)}

\newpage
\section{Conclusion}

\end{document}
